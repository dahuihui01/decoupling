\section*{Lecture 6}
We continue our discussion about multilinear restriction, and we gave a sketch of the proof for the second version, now we turn to the first version. (remember, the restriction conjecture concerns with bounding the size of $\|E\phi\|_{L^p}$ using $\|\phi\|$).

We now look at the idea of ``tiling'' going from the fourier space to the physical space. If we know a function whose fourier transform is supported on a region, can we disect the region, and look at the dual rectangles/tubes, and how they intersect, to bound the size of the original function $f$. To achieve, we need to familiarize ourselves with the idea of induction on scale.

Let's reiterate our multilinear restriction theorem.
\begin{theorem}[Multilinear restriction]
\begin{equation*}
    \left\|\prod_{j=1}^n |f_j|^{1/n} \right\|_{L_{avg}^p}(B_R)\leq R^\epsilon\prod_{j=1}^n\left(\|f_j\|_{L_{avg}^2(\omega B_R)} \right)^{1/n}
\end{equation*}
where $p=2n/(n-1)$.
\end{theorem}
Let's first denote $\Sigma_1, ..., \Sigma_n$ as hypersurfaces in $\R^n$, and then decompose $N_{1/R}\Sigma_j$ into disjoint unions os $R^{-1/2}$-caps $\theta$. Thus, we have $f_{j,\theta}$ as the restriction of $f$ whose fourier transform is supported on $\theta$ of $\Sigma_j$. The dimension of the $R^{-1/2}$ cap  $\theta$ is $R^{-1/2}\times R^{-1/2}\times...\times R^{-1/2}\times R$, so its dual tube $\theta^*$ has dimension $R^{1/2}\times...\times R^{1/2}\times R$. We know $|f_{j,\theta}$ is roughly constant on each $\theta^*$, as well as the translation of $\theta^*$, the tubes $T$'s. If we reexamine the multilinear restriction, then we get the LHS is estimating how the tubes overlap and the bound that we can obtain on the size the overlaps.

Now we introduce the induction on scales idea. If we consider a bigger region in the fourier space, that corresponds to a smaller region in the physical space. Hence, if we consider a fatter cap $R^{-1/4}$ cap $\tau$, and denote $f_\tau=\sum_{\theta\subset\tau}f_{j,\theta}$, then if we look at the dual $\tau^*$, it is a smaller tube enclosed in the intersection of $T$'s and has dimension $R^{1/4}\times...\times R^{1/4}\times R$. How $\tau^*$'s intersect tells us how energy is distributed/concentrated on these tubes.

We apply lcoal orthogonality to $\theta\subset\tau$. 
\begin{equation*}
    \|f_{j,\tau}\|_{L^2(B_{R_{1/2}}}^2\lesssim \sum_{\theta\subset\tau}\|f_{j,\theta}\|_{L^2(\omega)}^2
\end{equation*}
The reverse inequality for the local orthogonality is generality not true, if we write $f_{j,\theta}=\chi_\theta\widehat{f}_{j,\tau}$, then we have
\begin{equation*}
    f_{j,\theta}=\chi_\theta\ast f_{j,\tau}
\end{equation*}
if energy of $f_{j,\tau}$ outside $B_{R^{1/2}}$ may contribute a lot after the convolution. 
\begin{proposition}
    The inverse Fourier transform of characteristic functions $\check{\chi_\theta}$ doesn't decay fast. 
\end{proposition}
\begin{proof}
\begin{equation*}
    \check{\phi}(x)=\int\phi(\omega)e^{2\pi ix\omega}d\omega=\int_E\phi(\omega)e^{2\pi ix\omega}d\omega=\left.\frac{e^{2\pi ix\omega}}{2\pi ix}\right\vert_E
\end{equation*}
\end{proof}
\qed


