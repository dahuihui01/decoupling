\section*{Lecture 6}
We continue our discussion about multilinear restriction, and we gave a sketch of the proof for the second version, now we turn to the first version. (remember, the restriction conjecture concerns with bounding the size of $\|E\phi\|_{L^p}$ using $\|\phi\|$).

We now look at the idea of ``tiling'' going from the fourier space to the physical space. If we know a function whose fourier transform is supported on a region, can we disect the region, and look at the dual rectangles/tubes, and how they intersect, to bound the size of the original function $f$. To achieve, we need to familiarize ourselves with the idea of induction on scale.

Let's reiterate our multilinear restriction theorem.
\begin{theorem}[Multilinear restriction]
\begin{equation*}
    \left\|\prod_{j=1}^n |f_j|^{1/n} \right\|_{L_{avg}^p}(B_R)\leq R^\epsilon\prod_{j=1}^n\left(\|f_j\|_{L_{avg}^2(\omega B_R)} \right)^{1/n}
\end{equation*}
where $p=2n/(n-1)$.
\end{theorem}
Let's first denote $\Sigma_1, ..., \Sigma_n$ as hypersurfaces in $\R^n$, and then decompose $N_{1/R}\Sigma_j$ into disjoint unions os $R^{-1/2}$-caps $\theta$. Thus, we have $f_{j,\theta}$ as the restriction of $f$ whose fourier transform is supported on $\theta$ of $\Sigma_j$. The dimension of the $R^{-1/2}$ cap  $\theta$ is $R^{-1/2}\times R^{-1/2}\times...\times R^{-1/2}\times R$, so its dual tube $\theta^*$ has dimension $R^{1/2}\times...\times R^{1/2}\times R$. We know $|f_{j,\theta}$ is roughly constant on each $\theta^*$, as well as the translation of $\theta^*$, the tubes $T$'s. If we reexamine the multilinear restriction, then we get the LHS is estimating how the tubes overlap and the bound that we can obtain on the size the overlaps.

Now we introduce the induction on scales idea. If we consider a bigger region in the fourier space, that corresponds to a smaller region in the physical space. Hence, if we consider a fatter cap $R^{-1/4}$ cap $\tau$, and denote $f_\tau=\sum_{\theta\subset\tau}f_{j,\theta}$, then if we look at the dual $\tau^*$, it is a smaller tube enclosed in the intersection of $T$'s and has dimension $R^{1/4}\times...\times R^{1/4}\times R$. How $\tau^*$'s intersect tells us how energy is distributed/concentrated on these tubes.

We apply lcoal orthogonality to $\theta\subset\tau$. 
\begin{equation*}
    \|f_{j,\tau}\|_{L^2(B_{R_{1/2}}}^2\lesssim \sum_{\theta\subset\tau}\|f_{j,\theta}\|_{L^2(\omega)}^2
\end{equation*}
The reverse inequality for the local orthogonality is generality not true, if we write $f_{j,\theta}=\chi_\theta\widehat{f}_{j,\tau}$, then we have
\begin{equation*}
    f_{j,\theta}=\chi_\theta\ast f_{j,\tau}
\end{equation*}
if energy of $f_{j,\tau}$ outside $B_{R^{1/2}}$ may contribute a lot after the convolution. 
\begin{proposition}
    The inverse Fourier transform of characteristic functions $\check{\chi_\theta}$ doesn't decay fast. 
\end{proposition}
\begin{proof}
\begin{equation*}
    \check{\phi}(x)=\int\phi(\omega)e^{2\pi ix\omega}d\omega=\int_E\phi(\omega)e^{2\pi ix\omega}d\omega=\left.\frac{e^{2\pi ix\omega}}{2\pi ix}\right\vert_E
\end{equation*}
\end{proof}
\qed

The local orthogonality lemma lets us keep track of the size of the following expressions:
\begin{equation*}
    \sum|f_{j,\theta}(x)|^2, \sum|f_{j,\tau}(x)|^2, \sum_{bigger caps}|f_{j,\tau'}(x)|^2,..., \text{ until } \sum_{\Sigma_j}|f_{j,\Sigma_j}(x)|^2
\end{equation*}
We can think of $\sum|f_{j,\theta}(x)|^2$ as the energy density of $f_{j,\theta}$ and it concentrates on tubes ($T$) in the physical space. As we move to the right side, the tubes are getting finer and may move perpendicular to $x_j$-axis. Now we look at functions whose fourier support is precisely $\Sigma_j$.

Now let's define an extension operator over $\Sigma$.
\begin{definition}[Extension operator over $\Sigma$]
    For any smooth $\phi(x)\in C^\infty(\Sigma)$, we define an extension operator on $\Sigma$ of $\phi$ is 
    \begin{equation*}
        E_\Sigma\phi(x)=\int_\Sigma e^{2\pi i\omega x}\phi(x)dvol_\Sigma(\omega)
    \end{equation*}
\end{definition}

We restate the first version of the multilinear restriction, as well as its asusmptions, since they are important in our proof for the restriction conjecture in $n=2$ below. If $\tau$ are spherical caps of $\frac{1}{100n}$-nbd of $e_j$. Let $\phi_j: \Sigma_j\to\mathbb{C}$, and $f_j=E\phi_j$, then we have the following inequality.
\begin{theorem}[Multilinear restriction, first version]
\begin{equation*}
    \left\|\prod_{j=1}^n|E_{\Sigma_j}\phi_j|^{1/n} \right\|_{L^p(B_R)}\leq R^\epsilon\prod_{j=1}^n\|\phi_j\|_{L^2(\Sigma_j)}^{1/n}
\end{equation*}
where $p=2n/(n-1)$
\end{theorem}
Though not included in the restriction conjecture, we first prove the following:
\begin{proposition}
    \begin{equation*}
        \|E\phi\|_{L^2(B_R)}\lesssim R^{1/2}\|\phi\|_{L^2}
    \end{equation*}
\end{proposition}
\begin{proof}
This $L^2$ case is a lot simpler due to Plancherel. And we prove this by first proving the following lemma regarding integrating over hypersurfaces:
\begin{lemma}
    Let $\Pi$ be a hyperplane perpendicular to $e_j$, for some $j$, we then have
    \begin{equation*}
        \int_\Pi|E_j\phi|^2\sim\int_{\Sigma_j}|\phi|^2
    \end{equation*}
\end{lemma}
\begin{proof}
    We take $e_j$ to be $e_n$ for convenience. Then for $x\in\Pi$, we have $x=(x_1, ...,x_{n-1}, t)$ for fixed $t$ for all $x\in\Pi$. And $\Sigma_n$ is within a small neighborhood that is normal to $e_n$, hence we have for all $\omega\in\Sigma_n$, we have $\omega_n=h(\omega')=h(\omega_1,...\omega_{n-1})$. And because the integral operator integrates $dvol_{\Sigma_n(\omega)}$, let $J$ be the Jacobian determinant, we have
    \begin{equation*}
        dvol_{\Sigma_n}=Jd\omega'
    \end{equation*}
    And we would want to write $|E_j\phi|$ as a function or the fourier transform of a function to apply Plancherel. We have,
    \begin{equation*}
        E_{\Sigma_n}\phi(x)=\int_{\Sigma_n}e^{2\pi ix\omega}\phi(\omega)dvol_{\Sigma_n}(\omega)=\int_{\R^{n-1}}e^{2\pi ix'\omega'}e^{2\pi ith(\omega')}\phi(\omega')Jd\omega'
    \end{equation*}
    Then we define $g:\Sigma_n\to\mathbb{C}$ as $g(\omega')=e^{2\pi ith(\omega')}\phi(\omega')J$, then we have $E_{\Sigma_n}=\int e^{2\pi ix'\omega'}g(\omega')d\omega'=\check{g}$
    Hence by Plancherel, we have 
    \begin{equation*}
        \int_\Pi|E_{\Sigma_n}\phi|^2=\int_\Pi|\check{g}|^2=\int_\Pi|g|^2=\int_\Pi|\phi|^2J^2\sim\int_PiJ|\phi|^2=\int_{\Sigma_n}|\phi|^2dvol_{\Sigma_n}
    \end{equation*}
\end{proof}
\qed

And this lemma directly implies our bound on $\|E\phi\|_{L^2}$ since the only think left for us to do is to integrate over $x_n$, i.e. 
\begin{equation*}
    \|E\phi\|_{L^2}^2=\int_{-R}^R\int_\Pi|E_j\phi|^2\sim\int_{-R}^R\int_{\Sigma_n}|\phi|^2\leq R\|\phi\|_{L_2(\Sigma)}^2
\end{equation*}
Taking the square root gives us $\|E\phi\|_{L^2(B_R)}\lesssim R^{1/2}\|\phi\|_{L^2}$.
\end{proof}
\qed

We now restate our conjecture given our $p=2n/(n-1)$ from the restriction theorem. We expect
\begin{equation*}
    \|E\phi\|_{L^p(B_R)}\lesssim R^\epsilon\|\phi\|_{L^p(\Sigma)}
\end{equation*}
\begin{remark}
    Note for a general conjecture, one can put $L^q(\Sigma)$ on the RHS, for all $p\leq q\leq\infty$. But any case where $q<p$ would have a counterexample using a single wave packet.
\end{remark}
We will now prove the case where $n=2, p=4$.
\begin{theorem}
    For $n=2$, and thus $p=4$, let $\Sigma=S^1$, the unit circle in $\R^2$. We have, 
    \begin{equation*}
        \|E\phi\|_{L^4(B_R)}\lesssim R^\epsilon\|\phi\|_{L^4}
    \end{equation*}
\end{theorem}
\begin{proof}
    This is a long one and we will introduce lemmas and definitions along the way, here we go. We first state how may one use the multilinear restriction theorem. Let $\Sigma=\bigcup\tau$, where $\tau$ are $1/K$ caps, and there are roughly $K$ $\tau$'s.
    \begin{equation*}
        \int|E\phi|^p=\int|\sum_\tau E\phi_\tau|^p=\int\prod_{j=1}^n|\sum_\tau E\phi_\tau|^{p/n}\leq\int\prod_{j=1}^n\sum_\tau|E\phi_\tau|^{p/n}K^{O(1)}=K^{O(1)}\sum_\tau\int\prod_{j=1}^n|E\phi_\tau|^{p/n}
    \end{equation*}
    The term on the RHS is where we can apply our multilinear restriction theorem, and we have to make sure our $\tau$'s are small neighborhoods of $e_j$. We translate the above equation in terms of $n=2$. 
    \begin{equation*}
        \int|E\phi|^4\leq K^{O(1)}\sum_{\tau_1,\tau_2}\int|E\phi_{\tau_1}|^2|E\phi_{\tau_2}|^2
    \end{equation*}
    However, we can't always apply the multilinear restriction if we don't have our $\tau_1, \tau_2$ nicely close to $e_1, e_2$. Hence we categorize them and then deal with them separately.
    \begin{definition}[transverse caps]
        We say $(\tau_1, ..., \tau_n)$ are transverse if there is a linear change of variables $L$, with $\det(L)\lesssim K^{O(1)}$, such that $(L\tau_1,...., L\tau_n)$ are $\frac{1}{100n}$ neighborhoods of $e_j$.
    \end{definition}
    \begin{remark}
        A sequence $(\tau_1, ..., \tau_n)$ are not transverse if they all lie within $O(\frac{1}{k})$ neighborhood of the equator on $S^{n-1}$. In $n=2$, this means $\tau_1, \tau_2$ are transverse if they don't lie``directly across from each other'' on the unit circle.   
    \end{remark}
    Recall, we decompose $S^1=\bigsqcup\tau$ into $K^{-1}$ caps where $K\sim\log(R)$ and hence there are $\sim K$ caps. We pick out the $\tau$'s that contribute a lot at any given point $|E\phi(x)|$. And that would be 
    \begin{equation*}
        S(x)=\{\tau:|E\phi_\tau(x)|\geq\frac{1}{100K}|E\phi(x)|\}
    \end{equation*}
    Then we have, 
    \begin{equation*}
        \sum_{\tau\not\in S(x)}|E\phi_\tau(x)|\leq\frac{1}{100K}|E\phi_\tau(x)|K\leq\frac{1}{10}|E\phi(x)|
    \end{equation*}
    This means $\left|\sum_{\tau\in S(x)}|E\phi_\tau| \right|\geq\frac{9}{10}|E\phi(x)|$, hence we have
    \begin{equation*}
        \left|\sum_{\tau\in S(x)}|E\phi_\tau(x)| \right|\sim|E\phi(x)|
    \end{equation*}
Now we just need to look at $\tau$'s for each $x$ that are in $S(x)$. 
\begin{definition}[Broad and narrow points]
    We call $x$ is broad, if there exists $\tau_1, \tau_2$ such that $(\tau_1, \tau_2)$ are transverse and they are narrow if they are not transverse.    
\end{definition}
    To evaluate the broad portion, we note that we can make the linear change of variable such that 
    \begin{equation*}
        \int_{{B_R}\cap Broad}|E\phi|^4\leq K^{O(1)}\sum_{\tau_1, \tau_2, transverse}\int|E\phi_{\tau_1}|^2|E\phi_{\tau_2}|^2\leq K^{O(1)}R^\epsilon\sum_{\tau_1,\tau_2}\|\phi_{\tau_1}\|_{L^2}^2\|\phi_{\tau_2}\|_{L^2}^2
    \end{equation*}
    We continue with another line:
    \begin{equation*}
        K^{O(1)}R^\epsilon\sum\|\phi_{\tau_1}\|_{L^2}^2\|\phi_{\tau_2}\|_{L^2}^2\sim K^{O(1)}R^\epsilon\left(\|\phi_{\tau_1}\|_{L^2}^2+\|\phi_{\tau_2}\|_{L^2}^2 \right)^2\sim K^{O(1)}R^\epsilon\|\phi\|_{L^2}^4
    \end{equation*}
    Hence we are done with the broad the $x$'s, in this case, we know all $\tau_1, \tau_2$ are right across from each other, i.e. all the $\tau$ are contained in the $O(\frac{1}{K})$ neighborhood of the equator, then we have $|S(x)|\lesssim 1$.
    Then we would like to estimate $\int_{B_R\cap narrow}=\int|E\phi|^4$.
    \begin{equation*}
        \int_{B_R\cap narrow}|E\phi|^4=\int|\sum_\tau E\phi_\tau|^4\leq\sum_\tau\int|E\phi_\tau|^4
    \end{equation*}
    Again, the last line is by Holder's inequality. We are not able to use multilinear restriction anymore, but we can use our powerful tool of induction on scale. Remember we'd like to have $\|\phi\|_{L^2}^4$ on the RHS, now the question is how would we get that from the RHS above. We establish the following inequality;
    \begin{equation*}
        \|E\phi\|_{L^4(B_R)}\leq C(R)\|\phi\|_{L^4}
    \end{equation*}
    This can be seen as follows:
    \begin{equation*}
        \int|E\phi|^4\leq|B_R|\|\phi\|_{L^\infty}^4\leq|B_R|\|\phi\|_{L^1}^4\leq |B_R||S^1|^{1/r}\|\phi\|_{L^4}^4
    \end{equation*}
    Hence we define $C(R)$ as the smallest integer that we can place in the inequality above, such that 
    \begin{equation*}
        \|E\phi\|_{L^4(B_R)}\leq C(R)\|\phi\|_{L^4}
    \end{equation*}
    Now again, we would like to show that $C(R)\lesssim R^\epsilon$. Clearly the above bound is not good enough, and we would like to inprove that based on induction on scale. This is done by the following lemma. We are looking at $E\phi$ on a smaller scale $\tau$.
    \begin{lemma}
        We connect $\|E\phi_\tau\|_{L^4
        (B_R)}$ with $\|\phi_\tau\|_{L^4}$
        \begin{equation*}
            \|E\phi_\tau\|_{L^4(B_R)}\lesssim C(R/K)\|\phi_\tau\|_{L^4}
        \end{equation*}
    \end{lemma}
    \begin{proof}
        We now look at the proof by change of variables. For each $\tau$ of dimension $K^{-1}\times K^{-2}$, we do a change of variable such that $\tilde{\tau}$ is of dimension $1\times 1$. Because we've scaled it up by $K$, the dimension of the dual rectangle is scaled down by $K^{-1}$, hence we look at $\|E\phi_{\tilde{\tau}}\|_{L^4(B_{R/K})}$. By definition, we have
        \begin{equation*}
            \|E\phi_{\tilde{\tau}}\|_{L^4(B_{R/K})}\leq C\left(\frac{R}{K}\right)\|\phi_{\tilde{\tau}}\|_{L^4}
        \end{equation*}
        The linear change of variables does not change the constant that we place here.
    \end{proof}
    \qed

    Now we go back to the previous step where we have $\sum_\tau\int|E\phi_\tau|^4$ on the RHS.
    \begin{equation*}
        \int_{B_R\cap narrow}|E\phi|^4\leq\sum_\tau\int|E\phi_\tau|^4\leq C\left(\frac{R}{K}\right)^4\sum_\tau\int|\phi_\tau|^4=C\left(\frac{R}{K}\right)^4\sum_\tau\|\phi_\tau\|_{L^4}^4\sim C\left(\frac{R}{K}\right)^4\|\phi\|_{L^4}^4
    \end{equation*}
Hence combining the narrow and broad part, we have 
\begin{equation*}
    C(R)\lesssim K^{O(1)}R^\epsilon+C\left(\frac{R}{K}\right)
\end{equation*}
By induction, we look at $C(R/K)$, remember the above $K$ comes from $K\sim\log(R)$, hence becomes $\log(R/K)^{O(1)}$ in this case,
\begin{equation*}
    C\left(\frac{R}{K}\right)\lesssim \left(\frac{R}{K}\right)^\epsilon
\end{equation*}
Hence we have
\begin{equation*}
    C(R)\leq C_2(K)R^\epsilon+C_2C_1\left(\frac{R}{K}\right)^\epsilon
\end{equation*}
Hence the $R^\epsilon$ term dominates, and is what we want for $C(R)$.
\end{proof}
\qed

We restate what we just proved.
\begin{equation*}
    \|E\phi\|_{L^4}\lesssim R^\epsilon\|\phi\|_{L^4}
\end{equation*}
