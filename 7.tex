\section*{Lecture 7}
We will prove a weaker version of the decoupling theorem for the paraboloid. First we consider the paraboloid $P=\{x\in\R^n:x_n=x_1^2+...x_n^2\}$, and let $\Omega$ denote the $1/R$-nbd of $P$, i.e. $\Omega=N_{1/R}P$, and let $\theta$ be $R^{-1/2}$-caps of decomposition of $\Omega$. We again denote the decoupling constant as $D_p(R)$. 
\begin{theorem}[Decoupling, weaker]
    Let $2\leq p\leq\frac{2n}{n-1}$, we then have
    \begin{equation*}
        \|f\|_{L^p}\lesssim R^\epsilon\left(\sum_\theta\|f_\theta\|_{L^p}^2 \right)^{1/2}
    \end{equation*}
\end{theorem}
We recall the stronger version, and the only difference is that now we push the exponent $p$ to be $2\leq p\leq\frac{2(n+1)}{n-1}$. The weaker version avoids more technicalities since $2n/(n-1)$ is the exponent in the multilinear restriction theorem. The proof is in three parts: induction on scales and multiscale tools; multilinear restriction implies multilinear decoupling; finally, the decoupling theorem from multilinear decoupling.

\subsection*{Multiscale tools}
We now state the two main tools that make our multiscale arguments feasible.
\begin{lemma}[Linear change of variables preserves $D_p$]
    Fix $L:\R^n\to\R^n$ a linear change of variables, then for any decomposition of $\Omega$, we have
    \begin{equation*}
        D_p(\Omega=\bigsqcup_\theta\theta)=D_p(L\Omega=\bigsqcup L\theta)
    \end{equation*}
\end{lemma}
\begin{proof}
    In other words, this means we can ``stretch'' these decompositions, if we decompose an area using boxes of size $l$, then if we double the size of area we are trying to do decoupling on, then we decompose using boxes of size $2l$ to preserve the decoupling constant. Fix $f$, we want to construct $\tilde{f}$ such that when $supp(\widehat{f})\subset\Omega$, then we have $supp(\widehat{\tilde{f}})\subset L \Omega$.
    \begin{equation*}
        \tilde{f}(x)=f((L^*)^{-1}x)
    \end{equation*}
    We thus have, for the Fourier transform support of $f$,
    \begin{equation*}
        \int e^{2\pi i\omega\cdot x}f((L^*)^{-1}x)=\int e^{2\pi i\omega\cdot L^*y}f(y)|\det(L^*)|dy=|\det(L^*)|\int e^{2\pi iL\omega\cdot y}f(y)dy=|\det(L^*)|\widehat{f}(L\omega)
    \end{equation*}
    Hence we've stretched the domain of $\widehat{f}$ from $\Omega$ to $L\Omega$.
    By definition of the decoupling constant,
    we have
    \begin{align*}
        \|f\|_{L^p}&=|\det(L^*)|^{-1/p}\|\tilde{f}\|_{L^p}\\
        &\leq |\det(L^*)|^{-1/p}D_p(L\Omega=\bigsqcup L\theta)\left(\sum_\theta\|\tilde{f}_\theta\|_{L^p}^2 \right)^{1/2}\\
        &\leq D_p(L\Omega=\bigsqcup L\theta)\left(\sum_\theta\|f_\theta\|_{L^p}^2 \right)^{1/2}
    \end{align*}
    Hence we have $D_p(\Omega=\bigsqcup\theta)\leq D_p(L\Omega=\bigsqcup L\theta)$. And if we start from the domain $L\Omega$ and let $f(x)=\tilde{f}(((L^{-1})^*)^{-1}x)$, then we get the reserve direction.
\end{proof}
\qed

As an immediate corollary to this, we have the following (rough) equality.
\begin{corollary}[Two decompositions]
    If we write $R=R_1\cdot R_2$, then define $\theta$ as $R^{-1/2}$ caps and $\tau$ as $R_1^{-1/2}$ caps, then $\tau$ are larger caps that enclose $\theta$ (or $\theta$ is their refinement), then we have the following:
    \begin{equation*}
        D_p(\tau=\bigsqcup_{\theta\subset\tau}\theta)=D_p(R_2)
    \end{equation*}
\end{corollary}
\begin{proof}
    $\tau$ are $R_1^{-1/2}$ caps, while $\theta$ are $R_1^{-1/2}R_2^{-1/2}$ caps, then if we first center $\tau$ are the origin, and multiply the coordinate by $R_1^{-1/2}$, then $\tau$ is linearly transformed roughly into $\Omega$, while $\theta$ have become caps of size $R_2^{-1/2}$. This is what exactly the corollary means. We define an explicit map as follows:
    \begin{equation*}
        L_i(\omega)=\begin{cases}
            R_1^{1/2}(\omega_i-\alpha_i), 1\leq i\leq n\\
            R_1(\omega_n-\alpha_n)-2\sum_{j=1}^{n-1}\alpha_j(\omega_j-\alpha_j), i=n
        \end{cases}
    \end{equation*}
\end{proof}
\qed

The following proposition is quite indepdent of the previous lemmas.
\begin{lemma}
    Suppose $R=R_1\cdot R_2$, we partition $\Omega$ first into $R_1^{-1/2}$ caps $\tau$, then refine them to $R_1^{-1/2}R_2^{-1/2}$ caps $\theta$. Then we have
    \begin{equation*}
        D_p(\Omega=\bigsqcup\theta)\lesssim D_p(\Omega=\bigsqcup\tau)\cdot D_p(R_2)
    \end{equation*}
    Or in another form, $D_p(R)\lesssim D_p(R_1)\cdot D_p(R_2)$.
\end{lemma}
\begin{proof}
    \begin{align*}
        \|f\|_{L^p}&\leq D_p(R_1)\left(\sum_\tau\|f_\tau\|_{L^p}^2 \right)^{1/2}\\
        &\lesssim D_p(R_1)D_p(R_2)\left(\sum\tau\sum{\theta\subset\tau}\|f_\theta\|_{L^p}^2 \right)^{1/2}\\
        &=D_p(R_1)D_p(R_2)\left(\sum_\theta\|f_\theta\|_{L^p}^2 \right)^{1/2}
    \end{align*}
\end{proof}
\qed

Here we conclude the section which connects different scales and we will see how they are used in the later sections.

\subsection*{Multilinear Decoupling}
Just like how we used multilinear restriction to prove the restriction conjecture in $n=2$, we will use a multilinear version of the decoupling theorem to prove the actual decoupling theorem. And note how the multilinear decoupling theorem uses the multilinear restriction theorem (where we used the critical exponent of $n=2n/(n-1)$).

We first define the multilinear decoupling theorem.
\begin{enumerate}
    \item $P_1,..., P_n\subset P$ are transverse.
    \item $\Omega_j=N_{1/R}P_j$
    \item $\Omega_j=\bigsqcup\theta_j$
    \item Let $MD(R)$ be the smallest constant such that $supp(\widehat{f}_j)\subset P_j$ and $f_j=\sum_{\theta}f_{j,\theta}$ such that the following inequality holds:
\end{enumerate}
\begin{equation*}
    \left\|\prod_{j=1}^n|f_j|^{1/n} \right\|_{L_{avg}^p(B_R)}\leq MD(R)\prod_{j=1}^n\left(\sum_{\theta\subset\Omega_j}\|f_{j,\theta}\|_{L^p(\omega B_R)}^{1/n} \right)^{\frac{1}{2}\cdot\frac{1}{n}}
\end{equation*}
We first derive a simple relationship between $MD(R)$ and $D(R)$.
\begin{proposition}[MD(R) is no larger than $D(R)$]
    If we decompose $P$ and $\Omega$ like defined above, then we have
    \begin{equation*}
        MD(R)\leq D(R)
    \end{equation*}
\end{proposition}
\begin{proof}
    This can be done using Holder's inequality. 
    \begin{equation*}
        \left\|\prod_{j=1}^n|f_j|^{1/n} \right\|_{L^p}\leq\prod_{j=1}^n\|f_j^{1/n}\|_{L^{pn}}=\prod_{j=1}^n\|f_j\|_{L^p}^{1/n}
    \end{equation*}
    And for the RHS, we have
    \begin{equation*}
        \prod_{j=1}^n\|f_j\|_{L^p}^{1/n}\leq\prod_{j=1}^n\left(D_p(R)\sum_{\theta\subset\Omega_j}\|f_{j,\theta}\|_{L^p}^2 \right)^{\frac{1}{n}\cdot\frac{1}{2}}
    \end{equation*}
    With $MD(R)$ being the smallest constant one could put there, we have $MD(R)\leq D(R)$.
\end{proof}
\qed

Next, we prove another claim that we will use in the next section, which is the multilinear decoupling constant is the constant that we want for the decoupling constant.
\begin{proposition}[$MD(R)$ is what we want]
    For $2\leq p\leq\frac{2n}{n-1}$, we have
    \begin{equation*}
        MD_{p,n}(R)\lesssim R^\epsilon
    \end{equation*}
\end{proposition}
\begin{proof}
    We will use Holder and multilinear restriction. First by definition, we need to have $\|\prod\|_{L^p}$ on the LHS, then by Holder, for the average $L^p$ norm, we can replace $p$ with $2n/(n-1)$, then using multilinear restriction, to get $L^2$ on the RHS and again using local orthoganlity on small $\Omega_j$, and then getting back to $L^p$ on the RHS using Holder. The $R^\epsilon$ term arises from the multilinear restriction and all other inequalities do not come with constants.
    Now we translate that into some math.
    \begin{align*}
        \left\|\prod_{j=1}^n|f_j|^{1/n} \right\|_{L^p}&\leq \left\|\prod_{j=1}^n|f_j|^{1/n} \right\|_{L^{2n/(n-1)}}\\
        &\lesssim R^\epsilon\prod_{j=1}^n\|f_j\|_{L^2}^{1/n}\\
        &\lesssim R^\epsilon\prod_{j=1}^n\left(\sum_{\theta\subset\Omega_j}\|f_{j,\theta}\|_{L^2}^2 \right)^{\frac{1}{n}\cdot\frac{1}{2}}\\
        &\leq R^\epsilon\prod_{j=1}^n\left(\sum_{\theta\subset\Omega_j}\|f_{j,\theta}\|_{L^p}^2 \right)^{\frac{1}{n}\cdot\frac{1}{2}}
    \end{align*}
    Here I've skipped some notations, the multilinear restriction gives us $L^2(\omega B_R)$, which aligns with the definition in multilinear decoupling constant.
\end{proof}
\qed

So far, we've moven the multilinear decoupling theorem, which states for $2\leq p\leq 2n/(n-1)$, we have
\begin{equation*}
    \left\|\prod_{j=1}^n|f_j|^{1/n} \right\|_{L^p(B_R)}\lesssim R^\epsilon\prod_{j=1}^n\left(\sum_{\theta\subset\Omega_j}\|f_{j,\theta}\|_{L^p}^2 \right)^{\frac{1}{n}\cdot\frac{1}{2}}
\end{equation*}
Now the only thing left for us to do would be to connect the previous two sections to derive the decoupling theorem for the original large paraboloid.

\begin{remark}
    In the full decoupling theorem, we extended the exponent and that is gained from two Holder's inequlaities used above.
\end{remark}

\subsection*{Decoupling Theorem}
Recall how we proved
\begin{equation*}
    \|E\phi\|_{L^4(B_R)}\lesssim R^\epsilon\|\phi\|_{L^4}
\end{equation*}
We separated the points that are ``broad'' and ``narrow'', and noted the multilinear restriciton applies when we are dealing with the broad, but not narrow points; and for the narrow points, we gain information about where the normal vector lies and use that to derive a bound using induction on scales.

We now introduce the main theorem that we will prove about decoupling (which follows from multilinear decoupling), and also directly implies the decoupling constant $D(R)\lesssim R^\epsilon$ by induction on scale and dimension.

\begin{theorem}[Bound on $D_{p,n}(R)$]
    For any $K\geq 1$, we have
    \begin{equation*}
        D_{p,n}(R)\lesssim K^{O(1)}MD_{p,n}(R)+D_{p,n-1}(K^2)D_{p,n}(R/K^2)
    \end{equation*}
\end{theorem}
Now we should how this implies $D_{p,n}(R)\lesssim R^\epsilon$.
\begin{lemma}
    Given the bound on $D_{p,n}$ above, we have $D_{p,n}(R)\lesssim R^\epsilon$.
\end{lemma}
\begin{proof}
    Taking $K\sim\log(R)$, we have
    \begin{equation*}
        D_{p,n}\left(\frac{R}{K}\right)\lesssim\log(R/K)^{O(1)}\left(\frac{R}{K}\right)^\epsilon+\left(\frac{R}{K}\right)^\epsilon\lesssim\left(\frac{R}{K}\right)^\epsilon
    \end{equation*}
    Hence in the equation above in the theorem, we have the dominating term being $R^\epsilon$.
\end{proof}
\begin{remark}
    I don't know what is going on here.
\end{remark}

Going back to the theorem, let us examine what we have: we have the first term coming from the braod region, where we have caps that can be arranged up to a factor of $K^{O(1)}$ to transverse positions. And the second term comes from the narrow region.

We now give two lemmas, one regarding the broad, and one about the narrow region, so that we know where we are heading towards. And we will show these two lemmas imply the main theorem on the bound on $D_{p,n}(R)$.
Let $\tau$ be the $K^{-1}$ caps, and $\Omega=\bigsqcup\tau$. We specify the $\tau$'s that we care about, which is defined as follows. Let $S(B)$ denote the significant set of $\tau$ for $B$, where $B=B_r$ is a ball of radius $r$ in the physical space. Recall $f=\sum_\tau f_\tau$,
\begin{equation*}
    S(B)=\{\tau: \|f_\tau\|_{L^p(B)}\geq\frac{1}{100 |\tau|}\|f\|_{L^p(B)}\}
\end{equation*}
And just like the $n=2$ case for the restriction conjecture, we can show that $f_B=\sum_{\tau\in S(B)}f_\tau$ has $\|f_B\|_{L^p}\sim\|f\|_{L^p}$.

Basically we are decomposing the large ball $B_R$ which we are integrating over into balls $B$ of radius $B=B_r$. Then we categorize these balls as broad and narrow like previously.
\begin{definition}[Broad and Even balls]
    We define a ball $B$ to be broad if for the signiciant $\tau$ set, $S(B)$, we can find $\tau_1, ..., \tau_n$ such that they are transverse. (note here the number is the dimension). If $B$ is not broad, it is narrow.
\end{definition}
We now state the main two lemmas for broad and narrow estimates.
\begin{lemma}[Broad]
    The broad region is composed of disjoint $B$ that are broad, denoted as $Broad=\bigsqcup_{B, Broad}B$, and we have
    \begin{equation*}
        \|f\|_{L^p(Broad)}\leq r^{O(1)}MD_{p,n}(R)\left(\sum_{\theta}\|f_\theta\|_{L^p}^2 \right)^{1/2}
    \end{equation*}
\end{lemma}
And for the narrow one. Here we deal with individual balls $B_r$ that are narrow.
\begin{lemma}[Narrow]
    For each $B=B_r$, we take $r=K^2$, we have
    \begin{equation*}
        \|f_B\|_{L^p(B)}\lesssim D_{p,n-1}(K^2)\left(\sum_{\tau\in S(B)}\|f_\tau\|_{L^p(B)}^2 \right)^{1/2}
    \end{equation*}
\end{lemma}
Although here the narrow lemma is stated for each individual balls $B$, we can combine them as in ``parallel decoupling'' to get the same decoupling constant for the entire narrow region. By parallel decoupling, since $D_{p,n-1}$ is the decoupling constant holds for all $B\subset Narrow$, we thus have $Narrow=\bigsqcup_{B,narrow}B$,
\begin{equation*}
    \|f\|_{L^p(Narrow)}\lesssim D_{p,n-1}(K^2)\left(\sum_{\tau\in S(B)}\|f_\tau\|_{L^p}^2 \right)^{1/2}
\end{equation*}
Now we show how the two equations regarding broad and narrow show the main theorem on $D_{p,n}(R)$.
\begin{proposition}
    The above two lemmas imply the main theorem.
\end{proposition}
\begin{proof}
    By observing the main theorem, we notice that we still need to convert our narrow estimate into $\sum_\theta\|f_\theta\|_{L^p}$
    And this can be done by thinking about $\tau$ as large caps and their refinements are $\theta$'s. Then previously we've shown, if $R=R_1R_2$, with $\tau$ being $R^{-1/2}$ caps, then our $R_1=K^2$, hence $D(R)\lesssim D(K^2)D(\frac{R}{K^2})$,
    \begin{equation*}
        \|f\|_{L^p}\lesssim D(K^2)D(R/K^2)\left(\sum_\theta\|f_\theta\|_{L^p}^2 \right)^{1/2}
    \end{equation*}
    And we know $D(K^2)\geq\|f\|_{L^p}\left(\sum_\tau\|f_\tau\|_{L^p}^2 \right)^{-1/2}$, hence plugging in, we have
    \begin{equation*}
        \|f\|_{L^p(Narrow)}\lesssim D_{p,n-1}(K^2)\left(\sum_\tau\|f_\tau\|_{L^p}^2 \right)^{1/2}\lesssim D_{p,n-1}(K^2)D(R/K^2)\left(\sum_\theta\|f_\theta\|_{L^p}^2 \right)^{1/2}
    \end{equation*}
\end{proof}
    Combining, we have
    \begin{align*}
        \int|f|^p&=\int_{Broad}|f|^p+\int_{Narrow}|f|^p\\
        &\leq K^{O(1)p}MD(R)^p\left(\sum_\theta\|f_\theta\|_{L^p}^2 \right)^{p/2}+D_{p,n-1}^p(K^2)D^p(R/K^2)\left(\sum_\theta\|f_\theta\|_{L^p}^2 \right)^{p/2}\\
        &=\left(K^{O(1)p}MD(R)^p+D_{p,n-1}^p(K^2)D^p(R/K^2)\right)\left(\sum_\theta\|f_\theta\|_{L^p}^2 \right)^{p/2}\\
        &\lesssim \left(K^{O(1)}MD(R)+D_{p,n-1}(K^2)D(R/K^2)\right)^p\left(\sum_\theta\|f_\theta\|_{L^p}^2 \right)^{p/2}
    \end{align*}
\qed

We start with the narrow estimate.
Again, we are trying to show,
\begin{equation*}
    \|f\|_{L^p}\lesssim D_{p,n-1}\left(\sum_\tau\|f_\tau\|_{L^p}^2 \right)^{1/2}
\end{equation*}
But the parallel decoupling lemma, it suffices to show
\begin{equation*}
    \|f\|_{L^p(B)}\lesssim D_{p,n-1}\left(\sum_\tau\|f_\tau\|_{L^p(B)}^2 \right)^{1/2}
\end{equation*}
Now we claim, it suffices to show the following lemma.
\begin{lemma}[Lemma for the Narrow estimate]
    For the narrow balls $B$, let $\Pi^*$ be the hyperplane such that $nor(\tau)$ is close to $\Pi^*$ for all $\tau\in S(B)$. Again, we set $f_B=\sum_{\tau\in S(B)}f_\tau$, and for any $\Pi$ parallel to $\Pi^*$, we have
    \begin{equation*}
        \|f\|_{L^p(B\cap\Pi)}\lesssim D_{p,n-1}(K^2)\left(\sum_\tau\|f_\tau\|_{L^p(\omega B\cap\Pi)}^2 \right)^{1/2}
    \end{equation*}
\end{lemma}
Assuming this lemma, we prove the narrow estimate for each $B$.
\begin{proof}
Here we go.
\begin{align*}
    \|f\|_{L^p(B)}^p&\sim\|f_B\|_{L^p(B)}^p\\
    &=\int_t\|f\|_{L^p(B\cap\Pi)}^pdt\\
    &\lesssim D^p\int\left(\sum_\tau\|f_\tau\|_{L^p(\omega B\cap\Pi)}^2 \right)^{p/2}dt\\
    &=D^p\int \left(\sum_\tau\|f_\tau\|_{L^p(\omega B\cap\Pi)}^2 \right)\cdot \left(\sum_\tau\|f_\tau\|_{L^p(\omega B\cap\Pi)}^2\right)^{\frac{p-2}{2}}dt\\
    &\leq D^p\sum_{\tau\in S(B)}\int\|f_\tau\|_{L^p(B\cap\Pi)}^2 \left(\sum_\tau\|f_\tau\|_{L^p(\omega B\cap\Pi)}^2\right)^{\frac{p-2}{2}}\\
    &\leq D^p\sum_{\tau\in S(B)}\|(\|f_\tau\|_{L^p(B\cap\Pi)}^2)\|_{L^{p/2}}\|(\sum_\tau\|f_\tau\|_{L^p}^2)^{(p-2/)/2}\|_{L^(p/(p-2))}\\
    &=D^p\sum_{\tau\in S(B)}\left(\int_t\|f_\tau\|_{L^p(B\cap\Pi)}^p\right)^{2/p} dt\cdot \left(\int_t\left(\sum\|f_\tau\|_{L^p(B\cap\Pi)}^2\right)^{p/2}\right)^{(p-2)/p}\\
    &\leq D^p\sum_{\tau\in S(B)}\left(\int_t\|f_\tau\|_{L^p(B\cap\Pi)}^p\right)^{2/p} dt\cdot \left(\int_t\|f_\tau\|_{L^p(B\cap\Pi)}^p\right)^{-2/p}\int_t\left(\sum\|f_\tau\|_{L^p(B\cap\Pi)}^2\right)^{p/2}\\
    &=D^p\left(\sum_\tau\|f_\tau\|_{L^p(B)}^2 \right)^{p/2}
\end{align*}
Taking the $p$th root, we get 
\begin{equation*}
    \|f\|_{L^p(B)}\lesssim D_{p,n-1}\left(\sum_\tau\|f_\tau\|_{L^p(B)}^2\right)^{1/2}
\end{equation*}
\end{proof}
\qed

Hence to show the narrow estimate, it suffices to show the lemma that was introduced.
\begin{remark}
    I feel like I don't know enough geometry to fully understand this lemma, will come back to this.
\end{remark}

We shall proceed to the Broad estimate.
