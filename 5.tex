\section*{Lecture 5}

It's okay. we can make this work.
We will see how one can use multilinear restriction kakeya conjecture towards the restriction theory. More specifically, we will try and see how one can say about the size of $f$ given that $supp(\widehat{f}\subset\Omega)$, and knowing the ``geometry'' of $\Omega$.

We use $f=E\phi$ to denote $f=\int e^{2\pi ix\omega}\phi(\omega)d\omega$, then if $f=E_\Omega\phi$, then $supp(\widehat{f})\subset\Omega$.

Let's go back to a previous example, where $\theta\subset S^{-1}$ is a $R^{-1/2}$-cap on the unit sphere in $\R^n$. Let $\phi_\theta$ be a smooth bump funciton on $\theta$, $\|\phi\|_{L^p}\sim|\theta|^{1/p}\sim R^{\frac{-(n-1)}{2p}}$, then we have
\begin{equation*}
    |E\phi_\theta(x)|\sim|\theta|\sim R^{\frac{-(n-1)}{2}} \text{ on } \theta^*
\end{equation*}
Hence
\begin{equation*}
    \|E\phi_\theta\|_{B_R}\sim R^{\frac{-(n-1)}{2}+\frac{n+1}{2p}}
\end{equation*}

We will now consider the entire $S^{-1}$ and dissect it into spherical caps $\theta$ of size $R^{-1/2}$ as above, then the number $\theta$'s is $R^{\frac{n-1}{2}}$. Then let $\phi$ be constant 1 on $S^{-1}$, and let $\phi=\sum_\theta\phi_\theta$ (each $\phi_\theta)$ is the wave packet like above. If we let $B$ denote where every $\theta^*$ intsersect and have small frequency change, then $|E\phi|\sim\sum_\theta|E\phi_\theta|\sim R^{-(n-1)/2}\cdot R^{(n-1)/2}=1$ on $B$, but outside of $B$ and places where there are many intersections of $\theta^*$'s, say, $B_R\setminus B_{R/2}$, we have $|E\phi(x)|\sim|\theta|=R^{-(n-1)/2}$ on $B_R\setminus B_{R/2}$. To sum up, $|E\phi_\theta|\sim 1$ where all $\theta^*$'s intsersect and die off as you move away from the center. Hence the larger the $p$, the more focus on the higher values, hence we have, by interpolation,
\begin{equation*}
    \|E\phi\|_{L^p(B_R)}\sim\begin{cases}
        1, \text{ for } p>\frac{2n}{n-1}\\
        R^{\frac{-(n-1)}{2}}|B_R|^{1/p}=R^{\frac{-(n-1)}{2}-\frac{n}{p}}, p<\frac{2n}{n-1}
    \end{cases}
\end{equation*}
What does the above example of wave packets on $S^{n-1}$ tell us? Recall $\|\phi_\theta\|_{L^q}\sim R^{\frac{-(n-1)}{2q}}$, and the conjecture is that we can control the size of $\|E\phi\|_{L^p(B_R)}$ using the size of $\|\phi\|_{L^q}$. The conjecture is as follows:
\begin{equation*}
    \frac{\|E\phi\|_{L^p(B_R)}}{\|\phi\|_{L^q}}\lesssim R^\epsilon
\end{equation*}
For $q=2$, we have Tomas-Stein,
\begin{theorem}[Tomas-Stein]
    If $p\geq\frac{2(n+1)}{n-1}$, then $\|E\phi\|_{L^p(B_R)}\lesssim \|\phi\|_{L^2}$
\end{theorem}
And this gives one end of the range to put on $q$, and Stein conjectures the other end:
\begin{equation*}
    \|E\phi\|_{L^p(B_R)}\lesssim \|\phi\|_{L^\infty}, p>\frac{2n}{n-1}
\end{equation*}
Note: Only $n=2$ has been proven. Here we will us the wave packet approach. If we let $S^{n-1}=\bigcup\theta$, where $\theta$ are $R^{-1/2}$ caps, and let $\phi=\sum\phi_\theta$. By the Locally Constant Lemma, we have
\begin{equation*}
    |E\phi(x)|\sim \text{ constant on } T
\end{equation*}
where $T$ are translates of $\theta^*$. Hence, if we add things up ``horizontally'' like this, then we have
\begin{equation*}
    |E\phi(x)|\sim\sum_T a_T\chi_T
\end{equation*}
where $T$ are parallel to $\theta^*$ and are disjoint.

By Plancherel identity, we have
\begin{equation*}
    \|E\phi\|_{L^2}=\|\phi\|_{L^2}
\end{equation*}
which means
\begin{equation*}
    \sum a_T^2\sim\|\phi\|_{L^2}^2\leq\sum_\theta\|\phi_\theta\|_{L^2}^2\sim R^{\frac{n-1}{2}}\|\phi_\theta\|_{L^2}^2
\end{equation*}

To prove the decoupling, we have to look at how the tubes $T$ (which are translates of $\theta^*$) intersect. For each $B_{R^{1/2}}$, we define
\begin{equation*}
    \mu(B_R^{1/2})=\sum_{T\cap B_{R^{1/2}}\neq\emptyset}a_T^2
\end{equation*}
The Kakeya problem concerns with how many $B_{R^{1/2}}$ such that $\mu\sim 2^k$ for some $k$. The next thing we will do is to estimate $\|E\phi\|_{L^p(B_R)}$.

We now state two versions of multilinear restriction that we will give sketches/proofs for. Recall we define $f_j=E\phi_j$ to be the extension operator, i.e., taking the inverse Fourier transform.
\begin{theorem}[Multilinear Kakeya, first version]
    Let $\Sigma_j$ be spherical caps that are $\frac{1}{100n}$-neighborhood of $e_j$. Let $\phi_j:\Sigma_j\to\mathbb{C}$ and $f_j=E\phi_j$, then we have
    \begin{equation*}
        \|\prod_j|E\phi_j|^{\frac{1}{n}} \|_{L^p(B_R)}\lesssim R^\epsilon\prod_j\|\phi_j\|_{L^2}^{\frac{1}{n}}
    \end{equation*}
\end{theorem}
The first version of the restriction concerns with how to control the size of $E\phi_j$ using bounds on $\|\phi_j\|$. And now the second version of the multilinear problem concerns with how we control the size of $E\phi_j$ using $E\phi_j$.

\begin{theorem}[Multilinear restriction, second version]
    Suppose $\Sigma_1, \Sigma_2, ..., \Sigma_n$ are $C^2$ hyperspaces in $\R^n$, with diameter $\leq 1$ and $|curvature|\lesssim 1$, then if the angle between any the normal vector of $\omega_j\in\Sigma_j$ at $\omega_j$ and $e_j$ are $\leq\frac{1}{100n}$. Let $f_j=E\phi_j$, we have support $supp\widehat{f}_j\subset N_{1/k}\Sigma_j$. Then we have, for $p=\frac{2n}{n-1}$, 
    \begin{equation*}
        \|\prod_j|f_j|^{\frac{1}{n}}\|_{L^p}\lesssim R^\epsilon\prod_{j=1}^n\|f_j\|_{\sout{L}^2(\omega_{B_R})}^{\frac{1}{n}}
    \end{equation*}
\end{theorem}
We give the sketch of this using the lemmas that we already know. Consider a cover of $B_R$ using smaller balls of radisu $R^{1/2}$, i.e., $B_R=\bigcup\theta$, then by the local orthogonality lemma, we have
\begin{equation*}
    \|f_j\|_{L^p(B_{R^{1/2}})}^2\lesssim \sum_{\theta}\|f_{j,\theta}\|_{L^2(\omega_{B_{R_{1/2}}})}^2
\end{equation*}
And the ``white-lie'' version of this lemma would be to replace the inequality with roughly equal to.
\begin{equation*}
    \|f_j\|_{L^p(B_{R^{1/2}})}\sim\sum_\theta\|f_{j,\theta}\|_{L^2(\omega_{B_{R^{1/2}}})}
\end{equation*}
And we state something similar for the Locally Constant Lemma: let $T$ be a translation of $\theta^*$, for $x_1,x_2\in T$, we have
\begin{equation*}
    |f_{j,\theta}(x_1)|\sim|f_{j,\theta}(x_2)|
\end{equation*}

We now would like to prove the multilinear restriction, second version. Let's begin. Here we use $B$ to denote $B_R$ and $B_{1/2}$ to denote $B_{R^{1/2}}$.
\begin{proof}
    We will begin with the LHS, $\int\prod_j|f_j|^{1/n\cdot2n/(n-1)}$, we would like to have $R^\epsilon\prod_j\|f_j\|_{L^2}^{\frac{2}{n-1}}$ on the RHS.
    This we can do by using three key ingredientS: Local Constant Lemma, Local Orthogonality Lemma, and the Multilinear Kakeya estiamtes. Here we go,
    \begin{align*}
        \int\prod_j|f_j|^{\frac{2}{n-1}}&\sim |average|\int_{B_{1/2}}\prod_j|f_j|^{\frac{2}{n-1}}\\
        &\lesssim |a|R^{\frac{\epsilon n}{n-1}}\prod_j\|f_j\|_{L^\frac{2}{n-1}(B_{1/2})}\\
        &=R^\epsilon |a|\left(\int_{B_{1/2}}\prod_j|f_j|^2\right)^{1/(n-1)}\\
        &\sim |a|R^\epsilon\left(\int_{B_{1/2}}\prod_j\sum_\theta|f_{j,\theta}|^2\right)^{1/(n-1)}\\
        &\sim |a|R^\epsilon\int_{B_{1/2}}\prod_j\left(\sum_\theta|f_{j,\theta}|^2\right)^{1/(n-1)}\\
        &=R^\epsilon\int_B\prod_j\left(\sum_\theta|f_{j,\theta}|^2\right)^{(1/(n-1)}\\
        &\sim R^\epsilon\prod_j\left(\int_B\sum_\theta|f_{j,\theta}|^2\right)^{\frac{1}{n-1}}\\
        &\sim R^\epsilon\prod_j\left(\|f_j\|_{L^2}^2 \right)^{\frac{1}{n-1}}
    \end{align*}
    hence we are done!
\end{proof}