\section*{Lecture 4}

We first start with a warm up problem using projections. Consider a set $U$ in $\R^3$ is bounded,

$\begin{cases}
    Area(Proj_{xy-plane}(U))=A\\
    Area(Proj_{xz-plane}(U))=B\\
    Area(Proj_{yz-plane}(U))=C
\end{cases}$
Then what is $|U|$ bounded by?
\begin{proposition}
    \begin{equation*}
        |U|\leq (ABC)^{1/2}
    \end{equation*}
\end{proposition}
\begin{proof}
    We first just consider the projection of $U$ onto the $xy$-plane, i.e., fix $z$, we define $U_z=\{(x,y):(x,y,z)\in U\}$, then for any $z$, we have $|U_z|\leq A$. Then we cut up $U_z$ to find another bound using $B,C$. For $U_z$, fix $y$, if we define $X_z=\{x\in: (x,y)\in U_z\}$, and $Y_z=\{y\in\R: (x,y)\in U_z\}$. Hence by a simple picture, we have $|U_z|\leq|X_z||Y_z|$. Then we apply Cauchy-Schwarz:
    \begin{equation*}
        |U|\leq\int_z|U_z|dz\leq A^{1/2}(|X_z||Y_z|)^{1/2}\leq A^{1/2}\left(\int_z|X_z| \right)^{1/2}\left(\int_z|Y_z| \right)^{1/2}\leq(ABC)^{1/2}
    \end{equation*}
\end{proof}
As an immediate corollary, if we consider the boundary of $U$, because the projection to any plane is $\leq |\partial U|$ (as if you are mapping the whole surface out, which is ``maximal'' projection)
\begin{equation*}
    |U|\leq |\partial U|^{3/2}
\end{equation*}

\subsection{Tubes in different directions}
We first introduce the theorem that connects projections and incidence geometry.
\begin{theorem}[Loomis-Whitney]
    For $j=1,..., n$, define $f_j:\R^{n-1}\to R^+$, and $\pi_j: R^n\to\R^{n-1}$, and the projection by forgetting about the j-th coordinate, then we have
    \begin{equation*}
        \int_{\R^n}\prod_{j=1}^n(f_j\circ\pi_j)^{\frac{1}{n-1}}\leq\prod_{j=1}^n\left(\int_{R^{n-1}}f_j\right)^{\frac{1}{n-1}}
    \end{equation*}
\end{theorem}

Now we look at an example, $U\subset\R^n$ is a bounded region and $f_j=\chi_{\pi_j(U)}$ be the characteristic function of $\pi_j(U)$ and Loomis Whitney guarantees that we have,
\begin{equation*}
    |U|\leq\prod_{j=1}^n(|\pi_j(U)|)^{\frac{1}{n-1}}
\end{equation*}

$l_{j,a}\subset\R^n$ to be lines that are parallel to $x_j$-axis for $1\leq j\leq n$, and $1\leq a\leq N_j$. Let $T_{j,a}$ be the characteristic function of the 1-nbd of $l_{j,a}$. Surprisingly, we can provide some estimates on how these $T_{j,a}$ intersect.

\begin{proposition}[parallel case]
    Let everything be defined as above,i.e. when all lines are parallel to the axis, we have
    \begin{equation*}
        \int_{\R^n}\prod_{j=1}^n\left(\sum_{a=1}^{N_j}T_{j,a}\right)^{\frac{1}{n-1}}\lesssim \prod_{j=1}^nN_j^{\frac{1}{n-1}}
    \end{equation*}
\end{proposition}
\begin{proof}
    We need to define the $\pi_j, f_j$'s. Let $\pi_j$ be forgetting the $j$-th coordinate, and let $f_j=\sum_{a=1}^{N_j}\chi_{D_{j,a}}$, where $D_{j,a}$ denotes the disk of radius 1, with center where $l_{j,a}$ intersects $R^{n-1}$ with $j$-th axis removed $(l_{j,a}\cap\R^{n-1})$. Thus we have $\sum_{a=1}^{N_j}T_{j,a}=f_j\circ\pi_j$, hence we have
    \begin{equation*}
        \int_{\R^n}\prod_{j=1}^n\left(\sum_{a=1}^{N_j}T_{j,a}\right)^{\frac{1}{n-1}}=\int_{\R^n}\prod_{j=1}^n\left(f_j\circ\pi_j\right)^{\frac{1}{n-1}}\leq \prod_{j=1}^n\left(\int_{\R^{n-1}}f_j\right)^{\frac{1}{n}}=\prod_{j=1}^n\left(\int_{\R^{n-1}}\sum_{a=1}^{N_j}\chi_{D_{j,a}}\right)^{\frac{1}{n-1}}\lesssim \prod_{j=1^n}N_j^{\frac{1}{n-1}}
    \end{equation*}
\end{proof}
Where the last inequality follows from equality holds only when $D_{j,a}$ have no intersections/overlaps. 

\begin{example}
    Now we look at a cube of side length $s$, then for each direction $j$, its ``area'' is $s^{n-1}$, hence the number of tubes for each $j$ is $s^{n-1}$. And and 
    \begin{equation*}
        \sum_{a=1}^{N_j}T_{j,a}=1 \text{ on } E, 0 \text{ elsewhere }
    \end{equation*}
\end{example}
Hence we have the LHS of Loomis-Whitney inequality as
\begin{equation*}
    LHS\sim\int_{Q_s}1=s^n, RHS=\prod_{j=1}^nN_j^{\frac{1}{n-1}}=s^n
\end{equation*}
This means the inequality in the proposition is quite sharp.

Now we investigate the case where $l_{j,a}$ are not parallel to the $x_j$-axis, but rather are with an angle no larger than $\frac{1}{100n}$, and will the bound using Loomis-Whitney in the previous proposition still hold? With a simple picture, the answer is yes in $\R^2$.
\begin{equation*}
    \int_{\R^2}\sum_{a=1}^{N_1}T_{1,a}\sum_{b=1}^{N_2}T_{2,b}=\sum_{a=1}^{N_1}\sum_{b=1}^{N_2}\int_{\R^2}T_{1,a}\cdot T_{2,b}\lesssim N_1N_2
\end{equation*}
The last inequality follows from the fact that the product of $T_{1,a}\cdot T_{2,b}$ is equal to 1 in an area that may be slightly larger than 1 (going at different angles). The same inequality would hold if we replace lines with curves, and denote the angle at a given point $x\in\gamma_{j,a}$ to be the angle between the tangent line and the $x_j$-axis and have that less than $\delta$. However, the result is incorrect in $n=3$ for cuves, but there is a good result about tilting lines by Bennett-Carbery-Tao in 2005.

\begin{theorem}[Bennett-Carbery-Tao, 2005]
    Let $l_{j,a}$ be lines in $\R^n$ with an angle between $l_{j,a}$ and the $x_j$-axis no larger than $\frac{1}{100n}$. Let $T_{j,a}$ be the characteristic function of 1-neighborhood of $l_{j,a}$ (i.e. a tube of radius 1), then suppose $Q_s$ is a cube of side length $s$, then for any $\epsilon>0$, we have
    \begin{equation*}
        \int_{Q^s}\prod_{j=1}^n\left(\sum_{a=1}^{N_j}T_{j,a}\right)^{\frac{1}{n-1}}\lesssim s^\epsilon\prod_{j=1}^nN_j^{\frac{1}{n-1}}
    \end{equation*}
\end{theorem}
In other words, the inequality of Loomis-Whitney holds in $\R^n$ but with an $s^\epsilon$ loss.

We introduce the main lemma that is used to prove this theorem.
\begin{lemma}[Main lemma]
    For any $\epsilon>0$, there exists $\delta>0$ such that if all angles between $l_{j,a}$ and the $x_j$-axis are no larger than $\delta$, then we have
    \begin{equation*}
        \int_{Q^s}\prod_{j=1}^n\left(\sum_{a=1}^{N_j}T_{j,a}\right)^{\frac{1}{n-1}}\lesssim s^\epsilon\prod_{j=1}^nN_j^{\frac{1}{n-1}}
    \end{equation*}
\end{lemma}
We now prove how the main lemma implies the Theorem above. First we note that the angle between lines are no larger than $1/100n$, which is not our any $\delta>0$. But we can decompose the area on $S^{n-1}$ that are formed by points of lines going through the origin that are within $1/100n$ from the $x_j$-axis.

For each $j$, denote the area on the unit sphere $S^{-1}$ that is formed by points of lines $l_{j,a}$ of angle with $x_j$-axis no larger than $1/100n$, $S_j$, then we decompose $S_j$ into smaller portions of $S_j=\bigcup_b S_{j,b}$, where $S_{j,b}$ denotes the area that have diameter no larger than $\delta/10$ hence the angle that $S_j$ subtended is less than $\delta$. In $n=3$, there would be $S_1, S_2, S_3$. We further note that $e_j$ lives on $S^{n-1}$ and is the center of each $S_j$. Note that if for all $b$ $S_{j,b}$ is centered at $e_j$, then all lines are no more than $\delta$ from the $x_j$-axis, then we would have

If we denote $g_j=\sum_{a=1}^{N_j}T_{j,a}$, and denote $g_{j,b}$ as those if we pick out those in $S_{j,b}$, then $g=\sum_{b}g_{j,b}$.

Then in the theorem, we have
\begin{equation*}
    \int_{Q^s}\prod_{j=1}^n\left(\sum_{a=1}^{N_j}T_{j,a}\right)^{\frac{1}{n-1}}=\int_{Q^s}\prod_{j=1}^n\left(\sum_bg_{j,b}\right)^{\frac{1}{n-1}}\leq\sum_b\int_{Q^s}\prod_{j=1}^n g_{j,b}^{\frac{1}{n-1}}\lesssim \int_{Q^s}\prod_{j=1}^ng_{j,b}^{\frac{1}{n-1}}
\end{equation*}
The last inequality follows from the constant depends on $n, \delta$. Now to use the lemma, we note that if all $g_{j,b}$ are centered at $e_j$, then all lines are within $\delta$ of $x_j$-axis, but this is not the case. However, if one applies a linear transformation, the determinant is controlled by $\delta$, heence by an epsilon loss. Hence applying the lemma, we get
\begin{equation*}
    \int_{Q^s}\prod_{j=1}^ng_{j,b}^{\frac{1}{n-1}}\lesssim s^\epsilon\prod_{j=1}^nN_j^{\frac{1}{n-1}}
\end{equation*}

We now attempt to prove the main lemma now, and we do so by looking at a cube of size $s\leq\delta^{-1}$ and cutting it up into smaller cubes, and getting a good bound on the smaller ones, and finally rescaling $s\leq\delta^{-1}$ to $\leq\delta^{-m}$ for any large $m$. 

\begin{lemma}
    Let $s\leq\delta^{-1}$, then we have
    \begin{equation*}
        \int_{Q^s}\prod_{j=1}^n\left(\sum_{a=1}^{N_j}T_{j,a}\right)^{\frac{1}{n-1}}\lesssim \prod_{j=1}^nN_j(Q_s)^{\frac{1}{n-1}}
    \end{equation*}
    where $N_j(Q_s)=\{a: l_{j,a}\cap Q_s\neq\emptyset \}$, i.e. it is no larger than $N_j$.
\end{lemma}
\begin{proof}
    Since we can get a good bound on parallel case, we note that for each $T_{j,a}$, since $s\leq\delta^{-1}$, we can find $\tilde{T}_{j,a}$ suhc that $T_{j,a}\subset\tilde{T}_{j,a}$, and we use the same $\tilde{T}_{j,a}$ to denote its characteristic function. Hence we have,
    \begin{equation*}
        \int_{Q^s}\prod_{j=1}^n\left(\sum_{a=1}^{N_j}T_{j,a}\right)^{\frac{1}{n-1}}\leq \int_{Q^s}\prod_{j=1}^n\left(\sum_{a=1}^{N_j}\tilde{T}_{j,a}\right)^{\frac{1}{n-1}}\lesssim \prod_{j=1}^nN_j(Q_s)^{\frac{1}{n-1}}
    \end{equation*}
    where the last inequality follows from the parallel proposition.
\end{proof}

Then we consider smaller cube that are inside this cube. First we introduce the ``thickening'' of tubes, define $T_{j,a,w}$ to be the $w$ neighborhood of line $l_{j,a}$, namely, when $w=1, T_{j,a,1}=T_{j,a}$. This allows us to look at ``thick'' tubes and have them contain smaller ones.
\begin{lemma}
    Let $\frac{1}{20n}\delta^{-1}\leq s\leq\frac{1}{10n}\delta^{-1}$, then we have
    \begin{equation*}
        \int_{Q^s}\prod_{j=1}^n\left(\sum_{a=1}^{N_j}T_{j,a}\right)^{\frac{1}{n-1}}\lesssim_n \delta^n\int_{Q^s}\prod_{j=1}^n \left(\sum_{a=1}^{N_j}T_{j,a,\delta^{-1}} \right)^{\frac{1}{n-1}}
    \end{equation*}
\end{lemma}
\begin{proof}
    By the previous lemma, it suffices to show that 
    \begin{equation*}
        \prod_{j=1}^nN_j(Q_s)^{\frac{1}{n-1}}\lesssim \delta^n\int_{Q^s}\prod_{j=1}^n \left(\sum_{a=1}^{N_j}T_{j,a,\delta^{-1}} \right)^{\frac{1}{n-1}}
    \end{equation*}
    We note that if $s\leq\frac{1}{10n}\delta^{-1}$, then if $T_{j,a}\cap Q_s\neq\emptyset$, then we have $T_{j,a,\delta^{-1}}=1$ on $Q_s$, since $T_{j,a,\delta^{-1}}$ is simply a large, fat tube that contains $Q_s$. Hence we have, on $Q_s$, 
    \begin{equation*}
        N_j(Q_s)\leq\sum_{a=1}^{N_j}T_{j,a,\delta^{-1}}
    \end{equation*}
    Moreover, because $s\leq\frac{1}{10n}\delta^{-1}$, we have $|Q_s|\lesssim\delta^{-n}$, thus obtaining the inequality. And by the same reasoning, we have
    \begin{equation*}
        \int_{Q^s}\prod_{j=1}^n\left(\sum_{a=1}^{N_j}T_{j,a,\delta^{-(m-1)}}\right)^{\frac{1}{n-1}}\lesssim_n \delta^n\int_{Q^s}\prod_{j=1}^n \left(\sum_{a=1}^{N_j}T_{j,a,\delta^{-m}} \right)^{\frac{1}{n-1}}
    \end{equation*}
    And by inducting on $m$, we get
    \begin{equation*}
        \int_{Q^s}\prod_{j=1}^n\left(\sum_{a=1}^{N_j}T_{j,a}\right)^{\frac{1}{n-1}}\leq C(n)^m \delta^{nm}\int_{Q^s}\prod_{j=1}^n \left(\sum_{a=1}^{N_j}T_{j,a,\delta^{-m}} \right)^{\frac{1}{n-1}}
    \end{equation*}
\end{proof}

Now we connect the above two lemmas.
\begin{proposition}
    Let $s\leq\delta^{-1}$, then we have
    \begin{equation*}
        \int_{Q_s}\prod_{j=1}^n\left(\sum_{a=1}^{N_j}T_{j,a}\right)^{\frac{1}{n-1}}\leq C(n)\delta^n\int_{Q_s}\prod_{j=1}^n\left(\sum_{a=1}^{N_j}T_{j,a,\delta^{-1}} \right)^{\frac{1}{n-1}}
    \end{equation*}
\end{proposition}
\begin{proof}
This essentially follows from cutting $Q_s$ up into smaller cubes of the form in the previous lemma and use the lemma. Let $Q_s=\sum_bQ_{s,b}$,
\begin{equation*}
    \int_{Q_s}\prod_{j=1}^n\left(\sum_{a=1}^{N_j}T_{j,a}\right)^{\frac{1}{n-1}}=\sum_b\int_{Q_{s, b}}\prod_{j=1}^n\left(\sum_{a=1}^{N_j}T_{j,a}\right)^{\frac{1}{n-1}}\leq C(n)\delta^n\sum_b\int_{Q_{s,b}}\prod_{j=1}^n\left(\sum_{a=1}^{N_j}T_{j,a, \delta^{-1}}\right)^\frac{1}{n-1}
\end{equation*}
with the RHS $=C(n)\delta^n\int_{Q_s}\prod_{j=1}^n\left(\sum_{a=1}^{N_j}T_{j,a, \delta^{-1}}\right)^\frac{1}{n-1}$. And by the same reasoning, going along with what happened right before this proposition, for arbitrary large $m$, and a cube with $s\leq\delta^{-m}$, we have
\begin{equation*}
    \int_{Q_s}\prod_{j=1}^n\left(\sum_{a=1}^{N_j}T_{j,a}\right)^{\frac{1}{n-1}}\leq C(n)^m\delta^{nm}\int_{Q_s}\prod_{j=1}^n\left(\sum_{a=1}^{N_j}T_{j,a,\delta^{-m}} \right)^{\frac{1}{n-1}}
\end{equation*}
\end{proof}

Finally, we have all the tools to prove the main lemma, i.e. there always exists $\delta>0$ such that
\begin{equation*}
    \int_{Q_s}\prod_{j=1}^n\left(\sum_{a=1}^{N_j}T_{j,a}\right)^{\frac{1}{n-1}}\lesssim s^\epsilon\prod_{j=1}^nN_j^{\frac{1}{n-1}}
\end{equation*}
\begin{proof}
    Note that $\sum_{a=1}^{N_j}T_{j,a,\delta^{-m}}\leq N_j$, and $|Q_s|\leq \delta^{-nm}$, hence it suffices to choose $C(n)\leq s^\epsilon$. We note that
    \begin{equation*}
        C(n)^m\leq C(n)^{\frac{\log s}{-\log(\delta)}}=s^{\frac{\log C(n)}{-\log\delta}}
    \end{equation*}
    Hence it suffices to choose $\delta$ small enough such that $\frac{\log C(n)}{-\log\delta}\leq\epsilon$ as required.
\end{proof}
\qed