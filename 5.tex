\section*{Lecture 4}

We first start with a warm up problem using projections. Consider a set $U$ in $\R^3$ is bounded,

$\begin{cases}
    Area(Proj_{xy-plane}(U))=A\\
    Area(Proj_{xz-plane}(U))=B\\
    Area(Proj_{yz-plane}(U))=C
\end{cases}$
Then what is $|U|$ bounded by?
\begin{proposition}
    \begin{equation*}
        |U|\leq (ABC)^{1/2}
    \end{equation*}
\end{proposition}
\begin{proof}
    We first just consider the projection of $U$ onto the $xy$-plane, i.e., fix $z$, we define $U_z=\{(x,y):(x,y,z)\in U\}$, then for any $z$, we have $|U_z|\leq A$. Then we cut up $U_z$ to find another bound using $B,C$. For $U_z$, fix $y$, if we define $X_z=\{x\in: (x,y)\in U_z\}$, and $Y_z=\{y\in\R: (x,y)\in U_z\}$. Hence by a simple picture, we have $|U_z|\leq|X_z||Y_z|$. Then we apply Cauchy-Schwarz:
    \begin{equation*}
        |U|\leq\int_z|U_z|dz\leq A^{1/2}(|X_z||Y_z|)^{1/2}\leq A^{1/2}\left(\int_z|X_z| \right)^{1/2}\left(\int_z|Y_z| \right)^{1/2}\leq(ABC)^{1/2}
    \end{equation*}
\end{proof}
As an immediate corollary, if we consider the boundary of $U$, because the projection to any plane is $\leq |\partial U|$ (as if you are mapping the whole surface out, which is ``maximal'' projection)
\begin{equation*}
    |U|\leq |\partial U|^{3/2}
\end{equation*}

\subsection{Tubes in different directions}
We first introduce the theorem that connects projections and incidence geometry.
\begin{theorem}[Loomis-Whitney]
    For $j=1,..., n$, define $f_j:\R^{n-1}\to R^+$, and $\pi_j: R^n\to\R^{n-1}$, and the projection by forgetting about the j-th coordinate, then we have
    \begin{equation*}
        \int_{\R^n}\prod_{j=1}^n(f_j\circ\pi_j)^{\frac{1}{n-1}}\leq\prod_{j=1}^n\left(\int_{R^{n-1}}f_j\right)^{\frac{1}{n-1}}
    \end{equation*}
\end{theorem}

Now we look at an example, $U\subset\R^n$ is a bounded region and $f_j=\chi_{\pi_j(U)}$ be the characteristic function of $\pi_j(U)$ and Loomis Whitney guarantees that we have,
\begin{equation*}
    |U|\leq\prod_{j=1}^n(|\pi_j(U)|)^{\frac{1}{n-1}}
\end{equation*}

$l_{j,a}\subset\R^n$ to be lines that are parallel to $x_j$-axis for $1\leq j\leq n$, and $1\leq a\leq N_j$. Let $T_{j,a}$ be the characteristic function of the 1-nbd of $l_{j,a}$. Surprisingly, we can provide some estimates on how these $T_{j,a}$ intersect.

\begin{proposition}
    Let everything be defined as above, we have
    \begin{equation*}
        \int_{\R^n}\prod_{j=1}^n\left(\sum_{a=1}^{N_j}T_{j,a}\right)^{\frac{1}{n-1}}\lesssim \prod_{j=1}^nN_j^{\frac{1}{n-1}}
    \end{equation*}
\end{proposition}
\begin{proof}
    We need to define the $\pi_j, f_j$'s. Let $\pi_j$ be forgetting the $j$-th coordinate, and let $f_j=\sum_{a=1}^{N_j}\chi_{D_{j,a}}$, where $D_{j,a}$ denotes the disk of radius 1, with center where $l_{j,a}$ intersects $R^{n-1}$ with $j$-th axis removed $(l_{j,a}\cap\R^{n-1})$. Thus we have $\sum_{a=1}^{N_j}T_{j,a}=f_j\circ\pi_j$, hence we have
    \begin{equation*}
        \int_{\R^n}\prod_{j=1}^n\left(\sum_{a=1}^{N_j}T_{j,a}\right)^{\frac{1}{n-1}}=\int_{\R^n}\prod_{j=1}^n\left(f_j\circ\pi_j\right)^{\frac{1}{n-1}}\leq \prod_{j=1}^n\left(\int_{\R^{n-1}}f_j\right)^{\frac{1}{n}}=\prod_{j=1}^n\left(\int_{\R^{n-1}}\sum_{a=1}^{N_j}\chi_{D_{j,a}}\right)^{\frac{1}{n-1}}\lesssim \prod_{j=1^n}N_j^{\frac{1}{n-1}}
    \end{equation*}
\end{proof}
Where the last inequality follows from equality holds only when $D_{j,a}$ have no intersections/overlaps. 

\begin{example}
    Now we look at a cube of side length $s$, then for each direction $j$, its ``area'' is $s^{n-1}$, hence the number of tubes for each $j$ is $s^{n-1}$. And and 
    \begin{equation*}
        \sum_{a=1}^{N_j}T_{j,a}=1 \text{ on } E, 0 \text{ elsewhere }
    \end{equation*}
\end{example}
Hence we have the LHS of Loomis-Whitney inequality as
\begin{equation*}
    LHS\sim\int_{Q_s}1=s^n, RHS=\prod_{j=1}^nN_j^{\frac{1}{n-1}}=s^n
\end{equation*}
This means the inequality in the proposition is quite sharp.

Now we investigate the case where $l_{j,a}$ are not parallel to the $x_j$-axis, but rather are with an angle no larger than $\frac{1}{100n}$, and will the bound using Loomis-Whitney in the previous proposition still hold? With a simple picture, the answer is yes in $\R^2$.
\begin{equation*}
    \int_{\R^2}\sum_{a=1}^{N_1}T_{1,a}\sum_{b=1}^{N_2}T_{2,b}=\sum_{a=1}^{N_1}\sum_{b=1}^{N_2}\int_{\R^2}T_{1,a}\cdot T_{2,b}\lesssim N_1N_2
\end{equation*}
The last inequality follows from the fact that the product of $T_{1,a}\cdot T_{2,b}$ is equal to 1 in an area that may be slightly larger than 1 (going at different angles). The same inequality would hold if we replace lines with curves, and denote the angle at a given point $x\in\gamma_{j,a}$ to be the angle between the tangent line and the $x_j$-axis and have that less than $\delta$. However, the result is incorrect in $n=3$ for cuves, but there is a good result about tilting lines by Bennett-Carbery-Tao in 2005.

\begin{theorem}[Bennett-Carbery-Tao, 2005]
    Let $l_{j,a}$ be lines in $\R^n$ with an angle between $l_{j,a}$ and the $x_j$-axis no larger than $\frac{1}{100n}$. Let $T_{j,a}$ be the characteristic function of 1-neighborhood of $l_{j,a}$ (i.e. a tube of radius 1), then suppose $Q_s$ is a cube of side length $s$, then for any $\epsilon>0$, we have
    \begin{equation*}
        \int_{Q^s}\prod_{j=1}^n\left(\sum_{a=1}^{N_j}T_{j,a}\right)^{\frac{1}{n-1}}\lesssim s^\epsilon\prod_{j=1}^nN_j^{\frac{1}{n-1}}
    \end{equation*}
\end{theorem}
In other words, the inequality of Loomis-Whitney holds with an 