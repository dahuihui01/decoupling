We consider the following conjecture:
\begin{equation*}
    |\{1\leq a_i, b_i\leq N: a_1^3+a_2^3+a_3^3=b_1^3+b_2^3+b_3^3\}|\lesssim N^{3+\epsilon}
\end{equation*}

This follows from the natural Strichartz estimate. 

We observe the following integral.
\begin{equation*}
    \frac{1}{2\pi}\int_0^{2\pi}\left|\sum_{a=1}^ne^{ia^3x}\right|^6dx
\end{equation*}
The RHS is equal to the number of solutions to the diophantine equation above.

\begin{align*}
    \left|\sum_{a=1}^ne^{ia^3x}\right|^6&=(\sum_{a=1}^ne^{ia^3x})(\sum_{a=1}^ne^{-ib^3x})\\
    &=\sum_{a_1, a_2, a_3, b_1, b_2, b_3}e^{ix(a_1^3+a_2^3+a_3^3-b_1^3-b_2^3-b_3^3)}
\end{align*}

Hence the integral is 0 if the diophatine is satisfied, and 0 otherwise. Hence the integral evaluates exactly the number of diophatine equation. 

We consider the following conjecture:
\begin{equation*}
    |\{1\leq a_i, b_i\leq N: a_1^3+a_2^3+a_3^3=b_1^3+b_2^3+b_3^3\}|\lesssim N^{3+\epsilon}
\end{equation*}

This follows from the natural Strichartz estimate. 

We observe the following integral.
\begin{equation*}
    \frac{1}{2\pi}\int_0^{2\pi}\left|\sum_{a=1}^ne^{ia^3x}\right|^6dx
\end{equation*}
The RHS is equal to the number of solutions to the diophantine equation above.

\begin{align*}
    \left|\sum_{a=1}^ne^{ia^3x}\right|^6&=(\sum_{a=1}^ne^{ia^3x})(\sum_{a=1}^ne^{-ib^3x})\\
    &=\sum_{a_1, a_2, a_3, b_1, b_2, b_3}e^{ix(a_1^3+a_2^3+a_3^3-b_1^3-b_2^3-b_3^3)}
\end{align*}

Hence the integral is 0 if the diophatine is satisfied, and 0 otherwise. Hence the integral evaluates exactly the number of diophatine equation. 

\section*{Introduction to decoupling}
Now we move to the overview of decoupling.

If we denote a region $\Omega$ of $\R^n$ as the Fourier space, and we decompose it into small regions $\Omega=\bigsqcup\theta$.

If we assume the function $f$ whose Fourier transform has support in the region $\Omega$, then we can decompose $\Omega$, we will now make the definition as follows.
\begin{definition}[Decoupling]
    Let $f$ be a sufficiently regular function whose $supp(\widehat{f})\subset\Omega$, if we define
    \begin{equation*}
        f_\theta=\int_\theta\widehat{f}(\omega)e^{ix\omega}d\omega
    \end{equation*}
    Then by Fourier inverse formula, we get
    \begin{equation*}
        f=\sum_\theta f_\theta
    \end{equation*}
\end{definition}
\begin{proof}
    $\sum_\theta f\theta=\int_\Omega \widehat{f}(\omega)e^{ix\omega}d\omega=f(x)$
\end{proof}

One would like to control the norm $||f||_{L^p}$, using what you know about $||f_\theta||_{L^p}$. To give a general idea what we are heading towards, we can fine a constant $D_p$, dependent on $\Omega, \theta$, such that the following inequality is achieved.
\begin{equation*}
    ||f||_{L^p}(\R^n)\leq D_p(\Omega=\bigsqcup\theta)\left(\sum_\theta||f_\theta||_{L^p}^2\right)^{1/2}
\end{equation*}
Hence if we fix a specific decoupling choice, i.e. $\Omega=\bigsqcup_\theta\theta$, we can find an absolute constant such that
\begin{equation*}
    ||f||_{L^p}\leq C\left(\sum_\theta ||f_\theta||_{L^p}^2\right)^{1/2}
\end{equation*}
And we define the decoupling constant $D_p(\Omega=\bigsqcup\theta)$ to be the smallest of all $C$, for a fixed decoupling choice.



\begin{remark}
    Are there different ways of decomposing omega? Is the most intuitive way of decomposing the Fourier space based on frequency?
\end{remark}

\begin{proposition}[Estimate of sum]
    Let $R>0$, and pick Fourier space decomposition $\Omega=\bigsqcup_j\theta_j$, and let $g=\sum_ja_je^{i\omega_j\cdot x}$. If $B_{1/R}(\omega_j)\subset\theta_j$, for all $j$, then for any ball of radius $R$, $B_R$, we have
    \begin{equation*}
        ||g||_{L^p(B_R)}\lesssim D_p\left(\sum_j|a_j|^2 \right)^{1/2}R^{1/p}
    \end{equation*}
\end{proposition}
\begin{proof}
    Let $f=\eta g$, such that $supp(\widehat{\eta})\subset B_{1/R}$, such that $|\eta|\sim 1$ on $B_R$ and decays rapidly outside of $B_R$.
\end{proof}

