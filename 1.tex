We consider the following conjecture:
\begin{equation*}
    |\{1\leq a_i, b_i\leq N: a_1^3+a_2^3+a_3^3=b_1^3+b_2^3+b_3^3\}|\lesssim N^{3+\epsilon}
\end{equation*}

This follows from the natural Strichartz estimate. 

We observe the following integral.
\begin{equation*}
    \frac{1}{2\pi}\int_0^{2\pi}\left|\sum_{a=1}^ne^{ia^3x}\right|^6dx
\end{equation*}
The RHS is equal to the number of solutions to the diophantine equation above.

\begin{align*}
    \left|\sum_{a=1}^ne^{ia^3x}\right|^6&=(\sum_{a=1}^ne^{ia^3x})(\sum_{a=1}^ne^{-ib^3x})\\
    &=\sum_{a_1, a_2, a_3, b_1, b_2, b_3}e^{ix(a_1^3+a_2^3+a_3^3-b_1^3-b_2^3-b_3^3)}
\end{align*}

Hence the integral is 0 if the diophatine is satisfied, and 0 otherwise. Hence the integral evaluates exactly the number of diophatine equation. 

