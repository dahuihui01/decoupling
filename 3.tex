\section*{Lecture 2}
Now we begin with some building blocks.

Suppose $\Omega=[0,N], \theta_j=[j-1. j], \Omega=\bigsqcup_{j=1}^N\theta_j$. And we ask the question, if we have $supp(\widehat{f})\subset[0,1]$, could $|f|$ look like several narrow peaks and almost 0 elsewhere?

We recall how we decouple the function $f$: for $supp(\widehat{f})\subset\Omega$, define $f_{\theta_j}=\int_{[j-1, j]}\widehat{f}(\omega)e^{i\omega x}d\omega$, then $f=\sum_jf_{\theta_j}$.

Now we remind ourselves of the height of $f$.
\begin{proposition}
    Let $f\in\mathcal{S}$ be such that $supp(\widehat{f})\subset[0,1]$, and we have
    \begin{equation*}
        ||f||_{L^\infty}\lesssim||f||_{L^1}
    \end{equation*}
\end{proposition}
\begin{proof}
    We define a cutoff function $\eta\in\mathcal{S}$ such that $\eta=1$ on $[0,1]$, then $\widehat{f}=\eta\widehat{f}$, then $f=f\ast\check{\eta}$, also a Schwartz function.
\begin{align*}
    ||f||_{L^\infty}&=||f\ast\check{\eta}||_{L^\infty}\\
    &\leq||f||_{L^1}||\check{\eta}||_{L^\infty}\\
    &\lesssim||f||_{L^1}
\end{align*}
\end{proof}
Hence the answer is no, because if we have narrow peaks with controlled heights, $||f||_{L^1}$ would be small, which would violate $||f||_{L^\infty}\lesssim||f||_{L^1}$.

Now we ask the following question, can we have flat parts of $|f|$ where $||f||_{L^1}$ is dominated by the flat parts, but still has narrow peaks? To address that, we introduce an important lemma which allows us to control the height of $f$ in one interval using its $L^1$ norm in an even larger interval.

\begin{proposition}[Locally Constant Lemma]
    If $supp\widehat{f_1}\subset[0,1]$, and $I$ is the unit interval $[0,1]$, then we have
    \begin{equation*}
        ||f||_{L^\infty(I)}\lesssim ||f||_{L^1(\omega_I}
    \end{equation*}
    Where the weighted $L^1$ norm is defined to be $||f||_{L^1(\omega_I)}=\int_{\R}|f_1|\omega_I$
    where the function $\omega_I$ satisfies the following: $\omega_I\geq 0$, $\omega_I\sim 1$ on $I$, and $\omega_I$ decays rapidly off of $I$, lastly, $\omega_I$ is uniform in the sense that $\omega_{I+a}=\omega_I(\cdot-a)$
\end{proposition}
\begin{proof}
    This follows from the fact that $\eta\in\mathcal{S}$, hence $\check{\eta}\in\mathcal{S}$ as well, i.e. we have
    \begin{equation*}
        |\check{\eta}(y)|\lesssim \left(\frac{1}{1+|y|}\right)^{M}
    \end{equation*}
    for all large $M$.
Hence we follow the same computation:
\begin{align*}
    |f(x)|&=\left|\int f(y)\check{\eta}(x-y)dy \right|\\
    &\leq \int|f(y)||\check{\eta}(x-y)|dy\\
    &\leq \int|f(y)|\sup_{x\in I}|\check{\eta}(x-y)|dy
\end{align*}
And if we define $\omega_I(y)=\sup_{x\in I}|\check{\eta}(x-y)|$, surely it  satisfies being nonnegative, and by property of $\check{\eta}$ being Schwartz, $\check{\eta}\sim 1$ on $I$, and decays rapidly if $|x-y|$ is greater than 0.
\end{proof}
In other words, we almost know that 
$||f||_{L^\infty(I)}\lesssim ||f||_{L^1(2I)}$, where $2I$ is if we stretch the intervals keeping the same center.

\begin{remark}
    For $p=2, p=\infty$, the decoupling constant is easier to estimate. For $p=2$, we can apply Plancherel, namaley,
    \begin{equation*}
        ||f||_{L^2}=||\sum_jf_{\theta_j}||_{L^2}=||\sum_j\widehat{f}_{\theta_j}||_{L^2}=\sum_j||f_{\theta_j}||_{L^2}=\sum_{j}||f_{\theta_j}||_{L^2}
    \end{equation*}
    For $p=\infty$, we can apply Cauchy Schwartz, namely,
    \begin{equation*}
        ||f||_{L^\infty}=||\sum_jf_{\theta_j}||_{L^\infty}\leq\sum_j||f_{\theta_j}||_{L^\infty}\leq\left(\sum_j||f_{\theta_j}||_{L^\infty}^2\right)^{1/2}N^{1/2}
    \end{equation*}
\end{remark}

And now we conclude with an example. Consider a function $f_1$ with height 1, (i.e. $||f||_{L^\infty}=1)$ and $f_1(0)=1$ and it is concentrated on the interval $[-1,1]$. If we define $f_j(x)=e^{2\pi i(j-1)x}f_1(x)$ and define $f=\sum_jf_j$, then we have $f_j(0)=1$ and thus $f(0)=N$.

We note that $f_j$ oscillates with frequency $\frac{1}{j}$ and when $|x|\leq\frac{1}{10N}\leq\frac{1}{10j}$, we have $f_(x)\sim N$. Hence, if we consider $||f||_{L^p}$, we have
\begin{equation*}
    ||f||_{L^p}^p=\int|f|^p=
    \geq\int_{|x|\leq\frac{1}{10N}}|f|^p=
    \gtrsim\frac{1}{N}\cdot N^p=N^{p-1}
\end{equation*}
Hence taking the $1/p$ of both sides, we have $||f||_{L^p}\gtrsim N^{1-1/p}$.

Now if we would wish to consider the decoupling constant, we now consider $||f_j||_{L^p}$. Note $||f_j||_{L^p}\sim 1$, hence $\left(\sum_j||f_j||_{L^2}^2 \right)\sim N^{1/2}$. Thus we have $D_p\gtrsim N^{1/2-1/p}$.

\subsection*{Main Obstacle}
Consider a function $f_j$ such that $|f_j|=1$ on $[0,1]$, and is $\frac{1}{N}$ on $[1, N^3]$, and 0 elsewhere. Then the $||f_j||_{L^2}\sim N^{1/2}$, whereas $||f_j||_{L^4}\sim 1$. (Exactly how one owuld expect the $L^p$ norm to behave).

Like the above remark, we note that $||f||_{L^2}\sim \sum_j||f_j||_{L^2}^2)^{1/2}\sim N$, and $||f||_{L^\infty}\leq N^{1/2}(N)^{1/2}=N$. Now we ask the question, could $|f(x)|\sim N$ on the unit interval $[0,1]$?
The answer is no.
\begin{proof}
    Assume $|f(x)|\sim N$ on $[0,1]$, then $||f||_{L^4}\gtrsim N$, however, we know
    \begin{equation*}
        ||f||_{L^4}\lesssim D_p\left(\sum_j||f_j||_{L^4}^2\lesssim N^{1/4}\cdot N^{1/2}=N^{3/4} \right)
    \end{equation*}
    Note $D_p$ arises from our above lower bound given that $p=4$.
\end{proof}

Recall the Local Constant Lemma tells us how the height is controlled by the $L^1$ norm, now we introduce another lemma that connects the $L^2$ norms, which improves our estimate.
\begin{lemma}[Local Orthogonality Lemma]
    If $I$ is a unit interval, and $f=\sum_{j=1}^Nf_j$, and $supp\widehat{f_j}\subset[j-1,f]$, then we have
    \begin{equation*}
        ||f||_{L^2(I)}^2\lesssim\sum_j||f_j||_{L^2(\omega_I)}^2
    \end{equation*}
\end{lemma}
\begin{proof}
    We choose $\eta$ such that it preserves $f$ on the unit interval, and whose fourier transform has support land in $[-1,1]$, i.e. $|\eta|\sim 1$ on $I$, and $supp(\eta)\subset[-1,1]$.
    \begin{align*}
        ||f||_{L^2(I)}^2&=\int_I|f|^2\\
        &\leq\int_{\R}|\eta f|^2\\
        &=\int_{\R}|\widehat{\eta}\ast\widehat{f}|^2\\
        &=\int|\sum_j\widehat{\eta}\ast\widehat{f}|^2\\
        &\lesssim\sum_j\int_{\R}|\widehat{\eta}\ast\widehat{f}|^2\\
        &=\sum_j\int_{\R}|\eta|^2|f_j|^2\\
        &=\sum_j||f||_{L^2(\omega_I)}^2
    \end{align*}
    if we define $\omega_I=|\eta|^2$.
\end{proof}

We thus obtain this local orthogonality result, in the sense that we can decompose the $L^2$ norm locally and control the $L^2$ norm of $f$ by the sum of the $L^2$ norm of $f_j$.

Now we generalize this to a wide range of $p$ to obtain our local decoupling result.

\begin{proposition}[Local decoupling]
    If $I$ is a unit interval, for $2\leq p\leq\infty$, for each $1\leq j\leq N$, $supp(f_j)\subset[j-1,j]$, then we have
    \begin{equation*}
        ||f||_{L^p(I)}\lesssim N^{1/2-1/p}\left(\sum_{j=1}^N||f_j||_{L^p(\omega_I)}^2 \right)^{1/2}
    \end{equation*}
\end{proposition}
\begin{proof}
    This follows from the Locally Constant Lemma and the Locally Orthogonality Lemma above. 
    \begin{equation*}
        \int|f|^p=\int|f|^2|f|^{p-2}\leq||f||_{L^\infty(I)}^{p-2}\int|f|^2\leq\sum_{j}||f_{j}||_{L^2(\omega_I)}^2\left(\sum_j||f_j||_{L^\infty(I)}\right)^{p-2}
    \end{equation*}
\end{proof}
The last inequality follows from the local orthogonality lemma above which states $||f||_[L^2(I)]\leq\sum_j||f_j||_{L^2(\omega_I)}$.

Then for the second term, local constant lemma states that the height is controlled by the $L^1$ norm $||f_j||_{L^\infty}\lesssim ||f_j||_{L^1(\omega_I)}\lesssim ||f_j||_{L^2}$, where the last inequality is to match the $L^2$ norm of the first term. Combining, we have $\int|f|^p\leq (\sum_j\|f_j\|_{L^2(\omega_I)}^2)(\sum_j\|f\|_{L^2(\omega_I)})^{p-2}$. By Cauchy Schwarz on the second term, we obtain,
\begin{equation*}
    \int|f|^p\leq\left(\sum_{j}\|f_j\|_{L^2(\omega_I)}^2 \right)\left(\sum_{j}\|f_j\|_{L^2(\omega_I)}^2 \right)^{p/2-1}N^{p/2-1}=\left(\sum_{j}\|f_j\|_{L^2(\omega_I)}^2 \right)^{p/2}N^{p/2-1}
\end{equation*}
If we replace the $\|f\|_{L^2}$ with $\|f\|_{L^p}$, we get the desired result.

\begin{lemma}
    In finite measure spaces, for $p\geq q$, we have 
    \begin{equation*}
        \|f\|_{L^p}\lesssim \|f\|_{L^q}
    \end{equation*}
\end{lemma}
\begin{proof} This follows from Holder's inequality.
    \begin{equation*}
        \int_I|f|^p=\| |f|^p \|_{L^{q/p}}\mu(I)^{s}\lesssim\|f\|_{L^q}^p
    \end{equation*}
\end{proof}

Now we prove the parallel decoupling lemma, which basically states that if we decompose two measures as $\mu=\sum_i\mu_i, \omega=\sum_i\omega_i$, and for each $i$, we have the same decoupling constant, then we would be able to keep that decoupling constant when we sum them up. Recall the Minkowski's inequality refers to triangle inequality with respect to the $L^p$ norm.

\begin{proposition}[Parallel Decoupling Lemma]
    For some $p\geq 2$, and for any function $g=\sum_jg_j$, and any measures $\mu=\sum_i\mu_i, \omega=\sum_i\omega_i$, then if for each $i$, we have
    \begin{equation*}
        \|g\|_{L^p(\mu_i)}\leq D\left(\sum_j\|g_j\|_{L^p(\omega_i)}^2 \right)^{1/2}
    \end{equation*}
    then summing up, we would have the combined inequallity with the same decoupling constant,
    \begin{equation*}
        \|g\|_{L^p(\mu)}\leq D\left(\sum_j\|g_j\|_{L^p(\omega)}^2 \right)^{1/2}
    \end{equation*}
\end{proposition}
\begin{proof}
    The proof uses the Minkowski's inequality.
    \begin{align*}
        \int|g|^p\mu&=\sum_i\int|g|^p\mu_i\\
        &\leq D^p\sum_i\left(\sum_j\|g_j\|_{L^p(\omega_i)}^2 \right)^{p/2}\\
        &\leq D^p \left\|\sum_j||g_j||_{L^p(\omega_i)}^2 \right\|_{l_i^{p/2}}^{p/2}\\
        &\leq D^p\left(\sum_j\left\|||g_j||_{L^p(\omega_i)}^2 \right\|_{l_i^{p/2}}\right)^{p/2}\\
        &=D^p\left(\sum_j\left(\sum_i||g_j||_{L^p(\omega_i)}^p\right)^{2/p} \right)^{p/2}\\
        &=D^p\left(\sum_j||g_j||_{L^p(\omega)}^2 \right)^{p/2}
    \end{align*}
\end{proof}