\section*{Lecture 3}
We state Bourgain and Demeter's decoupling theorem for the parabaloid. We define a parabaloid as
\begin{equation*}
    P=\{\omega\in\R^n: \omega_n=\omega_1^2+...\omega_{n-1}^2, |\omega|\leq 1\}
\end{equation*}
We now define a slightly larger neighborhood of $P$, denoted by $N_{1/R}P$ as the neighborhood of $P$ of radius $1/R$ where $R$ is some large constant. This is the area that we will cut up using rectangles $\theta$,
\begin{equation*}
    \theta\approx R^{-1/2}\times R^{-1/2}\times...\times R^{-1}
\end{equation*}
Now we've fixed a decomposition, we denote its decoupling constant $D_p(R)=D_p(\omega=\cup\theta)$, and we ar eready to state our decoupling theorem.

\begin{theorem}[Bourgain and Demeter]
    For $2\leq p\leq\frac{2(n+1)}{n-1}$, for the ``cutting up'' scheme above, we have
    \begin{equation*}
        D_p(R)\lesssim R^{\epsilon}
    \end{equation*}
\end{theorem}

Now we define the dual of $\theta$, denoted by $\theta^*=\{x\in\R^n: \omega_\theta+\frac{1}{x}\in\theta\}$.  

nir is so stupid.